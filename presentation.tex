\documentclass{beamer}
\usepackage[utf8]{inputenc}
\usepackage[portuguese]{babel}

\title{Análise de Fatores Socioeconômicos que Impactam no Desempenho do Enem}
\author{Felipe Paradas}
\date{\today}

\begin{document}

\frame{\titlepage}

\section{Introdução}

\section{Fundamentação Teórica}

\section{Metodologia}

\subsection{Nutrição e Cuidados Médicos Básicos}
\begin{frame}{Nutrição e Cuidados Médicos Básicos}
\begin{itemize}
    \item \textbf{Cobertura Vacinal - Poliomielite}
    \begin{itemize}
        \item Fonte: IEPS, Sisvan/Ministério da Saúde.
        \item Definição: Percentual de crianças menores de 1 ano (vacina injetável) até 4 anos (oral) vacinadas.
    \end{itemize}
    \item \textbf{Hospitalizações por Condições Sensíveis à Atenção Primária}
    \begin{itemize}
        \item Fonte: IEPS, SIH, TabNet Datasus.
        \item Definição: Número de internações por 100.000 habitantes no SUS para condições evitáveis com atenção primária.
    \end{itemize}
    \item \textbf{Mortalidade Ajustada por Condições Sensíveis à Atenção Primária}
    \begin{itemize}
        \item Fonte: IEPS, SIM, TabNet Datasus.
        \item Definição: Taxa ajustada por idade de óbitos por 100.000 habitantes devido a CSAP.
    \end{itemize}
\end{itemize}
\end{frame}

\begin{frame}{Nutrição e Cuidados Médicos Básicos}
\begin{itemize}
    \item \textbf{Mortalidade Infantil até 5 anos}
    \begin{itemize}
        \item Fonte: Datasus/Ministério da Saúde.
        \item Definição: Óbitos de crianças menores de 5 anos por mil nascidos vivos.
    \end{itemize}
    \item \textbf{Subnutrição}
    \begin{itemize}
        \item Fonte: Sisvan/Ministério da Saúde.
        \item Definição: Percentual de indivíduos abaixo do peso ideal em todas as faixas etárias.
    \end{itemize}
\end{itemize}
\end{frame}

\subsection{Água e Saneamento}
\begin{frame}{Água e Saneamento}
\begin{itemize}
    \item \textbf{Abastecimento de Água via Rede de Distribuição}
    \begin{itemize}
        \item Fonte: CadÚnico/MDS.
        \item Definição: Percentual de domicílios com rede geral de distribuição.
    \end{itemize}
    \item \textbf{Esgotamento Sanitário Adequado}
    \begin{itemize}
        \item Fonte: CadÚnico/MDS.
        \item Definição: Percentual de domicílios com coleta de esgoto adequada.
    \end{itemize}
    \item \textbf{Índice de Abastecimento de Água}
    \begin{itemize}
        \item Fonte: SNIS/Ministério das Cidades.
        \item Definição: Percentual da população atendida com abastecimento de água.
    \end{itemize}
    \item \textbf{Índice de Perdas de Água na Distribuição}
    \begin{itemize}
        \item Fonte: SNIS/Ministério das Cidades.
        \item Definição: Percentual de água perdida na rede de distribuição.
    \end{itemize}
\end{itemize}
\end{frame}

\subsection{Moradia}
\begin{frame}{Moradia}
\begin{itemize}
    \item \textbf{Domicílios com Coleta de Resíduos Adequada}
    \begin{itemize}
        \item Fonte: CadÚnico/MDS.
        \item Definição: Percentual de domicílios com coleta direta pela concessionária.
    \end{itemize}
    \item \textbf{Domicílios com Iluminação Elétrica Adequada}
    \begin{itemize}
        \item Fonte: CadÚnico/MDS.
        \item Definição: Percentual de domicílios conectados à rede elétrica com medidor próprio.
    \end{itemize}
    \item \textbf{Domicílios com Paredes Adequadas}
    \begin{itemize}
        \item Fonte: CadÚnico/MDS.
        \item Definição: Percentual de domicílios com paredes de alvenaria ou madeira aparelhada.
    \end{itemize}
    \item \textbf{Domicílios com Pisos Adequados}
    \begin{itemize}
        \item Fonte: CadÚnico/MDS.
        \item Definição: Percentual de domicílios com pisos de cerâmica, carpete, cimento ou madeira aparelhada.
    \end{itemize}
\end{itemize}
\end{frame}

\subsection{Segurança Pessoal}
\begin{frame}{Segurança Pessoal}
\begin{itemize}
    \item \textbf{Assassinatos de Jovens}
    \begin{itemize}
        \item Fonte: Datasus/Ministério da Saúde e IBGE.
        \item Definição: Taxa de homicídios por 100.000 jovens entre 15 e 29 anos.
    \end{itemize}
    \item \textbf{Assassinatos de Mulheres}
    \begin{itemize}
        \item Fonte: Datasus/Ministério da Saúde e IBGE.
        \item Definição: Taxa de homicídios por 100.000 mulheres.
    \end{itemize}
    \item \textbf{Homicídios}
    \begin{itemize}
        \item Fonte: Datasus/Ministério da Saúde e IBGE.
        \item Definição: Taxa de homicídios em relação à população total.
    \end{itemize}
    \item \textbf{Mortes por Acidente de Transporte}
    \begin{itemize}
        \item Fonte: Datasus/Ministério da Saúde e IBGE.
        \item Definição: Taxa de mortes por acidentes de transporte por 100.000 habitantes.
    \end{itemize}
\end{itemize}
\end{frame}

\subsection{Acesso ao Conhecimento Básico}
\begin{frame}{Acesso ao Conhecimento Básico}
\begin{itemize}
    \item \textbf{Abandono no Ensino Fundamental}
    \begin{itemize}
        \item Fonte: Inep.
        \item Definição: Taxa de abandono escolar para alunos do ensino fundamental.
    \end{itemize}
    \item \textbf{Abandono no Ensino Médio}
    \begin{itemize}
        \item Fonte: Inep.
        \item Definição: Taxa de abandono escolar para alunos do ensino médio.
    \end{itemize}
    \item \textbf{Evasão no Ensino Médio}
    \begin{itemize}
        \item Fonte: Inep.
        \item Definição: Taxa de alunos que deixam de frequentar a escola de um ano para o outro.
    \end{itemize}
\end{itemize}
\end{frame}

\begin{frame}{Acesso ao Conhecimento Básico}
\begin{itemize}
    \item \textbf{Distorção Idade-Série no Ensino Médio}
    \begin{itemize}
        \item Fonte: Inep.
        \item Definição: Percentual de alunos com idade acima da esperada para a série cursada.
    \end{itemize}
    \item \textbf{Ideb Ensino Fundamental}
    \begin{itemize}
        \item Fonte: Inep.
        \item Definição: Índice que mede a qualidade do ensino fundamental, de 0 a 10.
    \end{itemize}
    \item \textbf{Reprovação Escolar no Ensino Médio}
    \begin{itemize}
        \item Fonte: Inep.
        \item Definição: Percentual de alunos reprovados no ensino médio.
    \end{itemize}
\end{itemize}
\end{frame}

\subsection{Acesso à Informação e Comunicação}
\begin{frame}{Acesso à Informação e Comunicação}
\begin{itemize}
    \item \textbf{Cobertura Internet Móvel (4G/5G)}
    \begin{itemize}
        \item Fonte: Anatel.
        \item Definição: Percentual de moradores com cobertura de internet móvel 4G/5G.
    \end{itemize}
    \item \textbf{Densidade de Internet Banda Larga Fixa}
    \begin{itemize}
        \item Fonte: Anatel.
        \item Definição: Número de acessos de banda larga fixa por 100 domicílios.
    \end{itemize}
    \item \textbf{Densidade de Telefonia Móvel}
    \begin{itemize}
        \item Fonte: Anatel.
        \item Definição: Número de acessos de telefonia móvel por 100 habitantes.
    \end{itemize}
    \item \textbf{Qualidade de Internet Móvel}
    \begin{itemize}
        \item Fonte: Anatel.
        \item Definição: Percentual de quedas ou congestionamentos na internet móvel.
    \end{itemize}
\end{itemize}
\end{frame}

\subsection{Saúde e Bem-estar}
\begin{frame}{Saúde e Bem-estar}
\begin{itemize}
    \item \textbf{Expectativa de Vida}
    \begin{itemize}
        \item Fonte: Ipea, IBGE e Pnud.
        \item Definição: Número médio de anos de vida esperados para um recém-nascido.
    \end{itemize}
    \item \textbf{Mortalidade entre 15 e 50 anos}
    \begin{itemize}
        \item Fonte: Datasus/Ministério da Saúde e IBGE.
        \item Definição: Número de óbitos por 100.000 habitantes na faixa etária de 15 a 50 anos.
    \end{itemize}
    \item \textbf{Mortalidade por DCNT}
    \begin{itemize}
        \item Fonte: Datasus/Ministério da Saúde e IBGE.
        \item Definição: Taxa ajustada por idade de óbitos por DCNT por 100.000 habitantes.
    \end{itemize}
\end{itemize}
\end{frame}


\begin{frame}{Saúde e Bem-estar}
\begin{itemize}
    \item \textbf{Obesidade}
    \begin{itemize}
        \item Fonte: Sisvan/Ministério da Saúde.
        \item Definição: Percentual da população em situação de obesidade segundo o IMC.
    \end{itemize}
    \item \textbf{Suicídios}
    \begin{itemize}
        \item Fonte: Datasus/Ministério da Saúde e IBGE.
        \item Definição: Taxa de mortalidade por suicídio por 100.000 habitantes.
    \end{itemize}
\end{itemize}
\end{frame}

\subsection{Qualidade do Meio Ambiente}
\begin{frame}{Qualidade do Meio Ambiente}
\begin{itemize}
    \item \textbf{Áreas Verdes Urbanas}
    \begin{itemize}
        \item Fonte: Mapbiomas.
        \item Definição: Percentual de área de vegetação sobre a área da mancha urbana.
    \end{itemize}
    \item \textbf{Emissões de CO$_2$e por Habitante}
    \begin{itemize}
        \item Fonte: SEEG e IBGE.
        \item Definição: Emissões de CO$_2$ equivalente por habitante (t CO$_2$e per capita).
    \end{itemize}
    \item \textbf{Focos de Calor}
    \begin{itemize}
        \item Fonte: Inpe e IBGE.
        \item Definição: Número de focos de calor por 10.000 habitantes.
    \end{itemize}
\end{itemize}
\end{frame}


\begin{frame}{Qualidade do Meio Ambiente}
\begin{itemize}
    \item \textbf{Índice de Vulnerabilidade Climática dos Municípios (IVCM)}
    \begin{itemize}
        \item Fonte: Instituto Votorantim.
        \item Definição: Índice de 0 a 100 que mede os riscos climáticos nos municípios.
    \end{itemize}
    \item \textbf{Supressão da Vegetação Primária e Secundária}
    \begin{itemize}
        \item Fonte: Mapbiomas.
        \item Definição: Percentual de área de vegetação suprimida no município.
    \end{itemize}
\end{itemize}
\end{frame}

\subsection{Direitos Individuais}
\begin{frame}{Direitos Individuais}
\begin{itemize}
    \item \textbf{Acesso a Programas de Direitos Humanos}
    \begin{itemize}
        \item Fonte: IBGE.
        \item Definição: Quantidade de programas municipais voltados aos direitos humanos.
    \end{itemize}
    \item \textbf{Existência de Ações para Direitos de Minorias}
    \begin{itemize}
        \item Fonte: IBGE.
        \item Definição: Verifica se há políticas públicas para grupos específicos.
    \end{itemize}
    \item \textbf{Índice de Atendimento à Demanda de Justiça}
    \begin{itemize}
        \item Fonte: CNJ.
        \item Definição: Percentual de processos baixados em relação aos casos novos.
    \end{itemize}
    \item \textbf{Taxa de Congestionamento Líquido de Processos}
    \begin{itemize}
        \item Fonte: CNJ.
        \item Definição: Percentual de processos não solucionados em relação ao total.
    \end{itemize}
\end{itemize}
\end{frame}

\subsection{Liberdades Individuais e de Escolha}
\begin{frame}{Liberdades Individuais e de Escolha}
\begin{itemize}
    \item \textbf{Acesso à Cultura, Lazer e Esporte}
    \begin{itemize}
        \item Fonte: IBGE.
        \item Definição: Existência de estruturas e eventos culturais e esportivos no município.
    \end{itemize}
    \item \textbf{Gravidez na Adolescência}
    \begin{itemize}
        \item Fonte: Datasus/Ministério da Saúde.
        \item Definição: Percentual de nascidos vivos com mães até 19 anos.
    \end{itemize}
    \item \textbf{Praças e Parques em Áreas Urbanas}
    \begin{itemize}
        \item Fonte: Mapbiomas e IBGE.
        \item Definição: Área de praças e parques urbanos por 10.000 habitantes.
    \end{itemize}
    \item \textbf{Trabalho Infantil}
    \begin{itemize}
        \item Fonte: CadÚnico/MDS.
        \item Definição: Número de casos de trabalho infantil por 10.000 famílias.
    \end{itemize}
\end{itemize}
\end{frame}

\subsection{Inclusão Social}
\begin{frame}{Inclusão Social}
\begin{itemize}
    \item \textbf{Paridade de Gênero na Câmara Municipal}
    \begin{itemize}
        \item Fonte: TSE.
        \item Definição: Índice de paridade de mulheres eleitas em relação à população feminina.
    \end{itemize}
    \item \textbf{Paridade de Negros e Pardos na Câmara Municipal}
    \begin{itemize}
        \item Fonte: TSE.
        \item Definição: Índice de paridade de negros e pardos eleitos em relação à população.
    \end{itemize}
    \item \textbf{Violência contra Indígenas}
    \begin{itemize}
        \item Fonte: Sinan-Datasus e IBGE.
        \item Definição: Número de casos de violência contra indígenas por 10.000 indígenas.
    \end{itemize}
\end{itemize}
\end{frame}


\begin{frame}{Inclusão Social}
\begin{itemize}
    \item \textbf{Violência contra Mulheres}
    \begin{itemize}
        \item Fonte: Sinan-Datasus e IBGE.
        \item Definição: Número de casos de violência contra mulheres por 100.000 mulheres.
    \end{itemize}
    \item \textbf{Violência contra Negros}
    \begin{itemize}
        \item Fonte: Sinan-Datasus e IBGE.
        \item Definição: Número de casos de violência contra negros por 100.000 pessoas negras.
    \end{itemize}
\end{itemize}
\end{frame}

\subsection{Acesso à Educação Superior}
\begin{frame}{Acesso à Educação Superior}
\begin{itemize}
    \item \textbf{Empregados com Ensino Superior}
    \begin{itemize}
        \item Fonte: Rais/MTE e IBGE.
        \item Definição: Número de empregos com nível superior por mil habitantes acima de 25 anos.
    \end{itemize}
    \item \textbf{Mulheres Empregadas com Ensino Superior}
    \begin{itemize}
        \item Fonte: Rais/MTE e IBGE.
        \item Definição: Número de mulheres empregadas com ensino superior por mil mulheres.
    \end{itemize}
\end{itemize}
\end{frame}

\section{Análise}

\section{Conclusão}


\end{document}
